\section{Introduction}
\label{sec:introduction}

Perceptrons are devices used to solve binary classification problems.
A perceptron tries to learn a hyperplane that divides the training examples in $2$ groups, depending on their label.
In general, if the dataset is linear separable, there are an infinite number of hyperplanes that perfectly separates the examples based on their labels.
The Minover algorithm \cite{minover} tries to find the hyperplane with the maximum stability, i.e. maximize the distance of the closest examples from the hyperplane.

In this assignment, we implement the Minover training algorithm and measured its generalization error on a linearly separable dataset as a function of the number of examples.
We also compare its performances with the Rosenblatt algorithm \cite{rosenblatt} implemented in the previous assignment.
Finally, we introduce different degrees of noise in the datasets and observe the performances of both algorithms.

\cref{sec:theory} introduces the Minover training algorithm.
\cref{sec:implementation} describes the implementation of the various experiments.
\cref{sec:evaluation} discusses the obtained results.
Finally, \cref{sec:conclusion} summarizes the most interesting results.
