\section{Introduction}
\label{sec:introduction}

Feed-forward neural networks are a useful tool to solve regression problems.
They consist in different levels of neurons (hidden units) called layers which have weighted directed connections with the following one.
Such networks normally have an input layer, the one that represents the data taken as an input by the network (with a number of neurons equal
to the dimension of the input), a variable number of layers, and therefore also hidden units, and finally an output layer, which in our case
corresponds to the sum of the values provided by the last hidden units layer. In this network the information is passed
just forward (i.e. to the following layer), in order to avoid loops, with the only exception of the  weights' update vectors, in which the gradient is calculated backward.
Networks builded in this way represent an advancement of the perceptron and can reach much better performances (as in the case of storage where the capacity of a feed forward network is $\alpha = K \sqrt{ln K}$).

\cite{chauvin1995backpropagation}
