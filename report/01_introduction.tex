\section{Introduction}
\label{sec:introduction}

In this work we will present a Matlab implementation of a Rosenblatt Perceptron \cite{rosenblatt1958perceptron}
and how we made use of it in order to obtain suitable data that let us observe how the theory about the capacity of the
hyperplane can be applied in a computer experiment. This experiment can also be seen as a study about how many
data can be stored in the Perceptron, even though this is not the aim of this mathematical construct.
In fact the Rosenblatt Perceptron is the first mathematical model of a biological neuron which is designed to
solve the problem of classifying an input by finding a linear separation in the context a set containing P n-dimensional input vectors.
This theoretical construct is able to provide such a solution by learning how to give the correct answer for a certain input
from a set of labeled data.

The theoretical limit of the linear separation problem for a random dataset $D^P_N$ in the case of a perceptron is $\alpha <= 2$, whith $\alpha = P/N$
(P = \#element in the learning dataset, N = \#features)



\begin{itemize}
\item general motivation for your work, context and goals.
\item problem: what is the problem you are trying to address, solve, or reason on?
\item strategy: the way you will address the problem
\end{itemize}

The introduction or first section can also have a different title if you think another one represents better the content. 

Don't write obvious/silly things like ``This is the report of Assignment No.\ 1,'' or cut \& paste here parts of the assignment. 
