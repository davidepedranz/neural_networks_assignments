\section{Implementation}
\label{sec:implementation}

TODO!

% \subsection{Dataset Generation}
% The first step of the experiment is to generate an artificial dataset with $P$ random N-dimensional feature vectors and binary labels $\mathcal{D} = \{ \xi^\mu, S^\mu \,|\, \xi^\mu \in \mathds{R}^N, S^\mu \in \{+1, -1\} \}_{\mu=1}^P$.
% The feature vectors $\xi^\mu$ have independent random components $\xi_j^\mu \sim \mathcal{N}(0, 1)$ and are generated using the \texttt{randn} command in Matlab.
% The labels $S^\mu$ are assuming the values $\{+1, -1\}$ with equal probability of $1/2$.

% \begin{lstlisting}[language=Matlab]
%   function [X, y] = generate_dataset(P, N)
%       X = randn(P, N);
%       y = iff(rand(P, 1) < 0.5, -1, 1);
%   end
% \end{lstlisting}

% \subsection{Perceptron Training}
% The next step is to use the Rosenblatt algorithm to implement and train a perceptron on the generated dataset.
% The Rosenblatt algorithm is a sequential procedure that learns the weights of the perceptron from the examples.
% The weights are initialized to zero, then the training examples are presented one by one to the perceptron for a given number of epochs (iterations).
% If the example is correctly classified according to the current weights, no update is performed;
% if the example if wrongly classified, the weights are updated according to the following rule:
% \begin{gather} 
%     w(0) = 0, \\
%     w(t + 1) =  \begin{cases}
%                     w(t) + \frac{1}{N} \xi^{\mu(t)} S^{\mu(t)} &\text{if $E^{\mu(t)} \leq 0$}\\
%                     w(t) &\text{else}
%                 \end{cases},
%     \label{eq:update-rule}
% \end{gather}
% where $E^{\mu(t)} = w(t) \xi^{\mu(t)} S^{\mu(t)}$, $t = 1, 2, ...$ represents the current time step and $\mu(t) = 1, 2, ..., P, 1, 2, ...$ identifies the current example.
% The algorithm is stopped either when $E^\mu > 0$ for all examples in the dataset or the maximum number of epochs $n_{max}$ is reached.

% \subsection{Experiments}
% For a fixed value of $P$ and $N$, $n_D$ independent datasets are generated.
% A new perceptron is trained on each dataset for at most $n_{max}$ epochs.
% For each dataset, we compute the rate fraction $Q_{l.s.}$ of successful runs, where a run is successful if the perceptron correctly classifies each example in the dataset at the end of the training phase.
% $Q_{l.s.}$ gives an estimate of the probability that the perceptron finds a linear separation in a random dataset of $P$ independent points in $N$ dimension.

% Multiple experiments are run for different values of $P$ and $N$ in order to compute $Q_{l.s.}$ as a function of $\alpha = P / N$.
% In other words, we study the probability of the perceptron to find a linear separation as a function of the rate between the number of examples and the dimension of the input.

% \subsection{Bonus}
% \subsubsection{Weight Update Criterion}
% We modified the function to train the perceptron to update the weights' vector $\mathsf{\bm{w}}$ if $E^\mu < c$ instead of $E^\mu < 0$ (see \cref{eq:update-rule}).
% Also, we changed the algorithm to stop the training only if $E^{\mu} > c$ for all examples in the dataset or the maximum number of epochs $n_{max}$ is reached.

% \subsubsection{Inhomogeneous Perceptrons}
% Rosenblatt perceptrons only learns separation hyperplanes that goes through the origin (homogeneous).
% In general, a solution to the classification problem may be a hyperplane that does not go through the origin (inhomogeneous), but be in the form:
% \begin{equation}
%     S = sign(\mathsf{\bm{w}} \cdotp \xi + \theta) = \pm 1
% \end{equation}

% It is possible to generalize the Rosenblatt perceptron to learn also inhomogeneous separation hyperplanes by adding an artificial dimension to the dataset and forcing it to a non-zero constant (e.g. $-1$) and increasing the size of the weights' vector $\mathsf{\bf{w}}$ by one (the last element of $\mathsf{\bf{w}}$ corresponds to the intercept $\theta$).
