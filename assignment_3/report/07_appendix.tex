\lstset{
    language=Matlab,%
    basicstyle=\fontsize{9}{11},
    breaklines=true,%
    morekeywords={matlab2tikz},
    keywordstyle=\color{blue},%
    morekeywords=[2]{1}, keywordstyle=[2]{\color{black}},
    identifierstyle=\color{black},%
    stringstyle=\color{mylilas},
    commentstyle=\color{mygreen},%
    showstringspaces=false,%without this there will be a symbol in the places where there is a space
    numbers=left,%
    numberstyle={\tiny \color{black}},% size of the numbers
    numbersep=0pt, % this defines how far the numbers are from the text
    linewidth=\columnwidth,
    emph=[1]{for,end,break},emphstyle=[1]\color{blue}, %some words to emphasize
}

\begin{appendices}
    \section{Matlab code}
    \subsection{Weights' Gradients}
    \label{sub:gradients}
    \begin{lstlisting}[language=Matlab] 
        function [g1, g2] = gd(example, tau, w1, w2)
            %GD Compute the gradients of the error function for the given example
            %with respect to w1 and w1 (weights of the 2 hidden units).
        
            tanh_w1 = tanh(example * w1);
            tanh_w2 = tanh(example * w2);
            sigma = tanh_w1 + tanh_w2;
            g_comp = (sigma - tau) * example';
            g1 = g_comp * (1 - (tanh_w1 ^ 2));
            g2 = g_comp * (1 - (tanh_w2 ^ 2));
        end
    \end{lstlisting}
\end{appendices}
