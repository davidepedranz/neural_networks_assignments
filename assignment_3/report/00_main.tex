\documentclass[conference]{IEEEtran}

\usepackage[utf8]{inputenc}
\usepackage[T1]{fontenc}
\usepackage{silence}\WarningsOff[latexfont]

\usepackage{listings}

\usepackage{amsmath}
\usepackage{bm}
\usepackage{dsfont}
\usepackage{amssymb}

\usepackage{graphicx}
\usepackage{cite}
\usepackage{url}
\usepackage[caption=false,font=footnotesize]{subfig}
\usepackage[binary-units,per-mode=symbol]{siunitx}
\sisetup{list-final-separator = {, and }}
\usepackage{booktabs}
\usepackage{pifont}
\usepackage{microtype}
\usepackage{textcomp}
\usepackage[american]{babel}
\usepackage[noabbrev,capitalise]{cleveref}
\usepackage{xspace}
\usepackage{hyphenat}
\usepackage{bm}
\usepackage[draft,inline,nomargin,index]{fixme}
\fxsetup{theme=color}
\usepackage{grffile}
\usepackage{xfrac}
\usepackage{multirow}

\usepackage{color} %red, green, blue, yellow, cyan, magenta, black, white
\definecolor{mygreen}{RGB}{28,172,0} % color values Red, Green, Blue
\definecolor{mylilas}{RGB}{170,55,241}
\RequirePackage{xstring}
\RequirePackage{xparse}
\RequirePackage[index=true]{acro}
\NewDocumentCommand\acrodef{mO{#1}mG{}}{\DeclareAcronym{#1}{short={#2}, long={#3}, #4}}
\NewDocumentCommand\acused{m}{\acuse{#1}}

\lstset{language=Matlab,%
%basicstyle=\color{red},
basicstyle=\fontsize{9}{11},
breaklines=true,%
morekeywords={matlab2tikz},
keywordstyle=\color{blue},%
morekeywords=[2]{1}, keywordstyle=[2]{\color{black}},
identifierstyle=\color{black},%
stringstyle=\color{mylilas},
commentstyle=\color{mygreen},%
showstringspaces=false,%without this there will be a symbol in the places where there is a space
numbers=left,%
numberstyle={\tiny \color{black}},% size of the numbers
numbersep=0pt, % this defines how far the numbers are from the text
linewidth=\columnwidth,
emph=[1]{for,end,break},emphstyle=[1]\color{blue}, %some words to emphasise
%emph=[2]{word1,word2}, emphstyle=[2]{style},    
}

\usepackage{blindtext}

\begin{document}

\title{
    Learning by gradient descent\\
    \large Neural Networks and Computational Intelligence - Practical Assignment III
}

\author{
    \IEEEauthorblockN{Samuel Giacomelli}
    \IEEEauthorblockA{\small Student Number: S3546330 \\ s.giacomelli@student.rug.nl}
    \and
    \IEEEauthorblockN{Davide Pedranz}
    \IEEEauthorblockA{\small Student Number: S3543757 \\ d.pedranz@student.rug.nl}
}

\maketitle

\begin{abstract}
    Feed-forward neural networks are powerful devices that can be used to solve regression problems.
    Since a network is formed by many units and the output is a continuous value, the algorithms designed for the perceptron can not be used for the training.
    A possible approach is to define a cost function on the training examples and minimize it using numerical optimization techniques.
    In this assignment, we implement Stochastic Gradient Descent to train a feed-forward network with 1 hidden layer of 2 units.
\end{abstract}

\acresetall

\section{Introduction}
\label{sec:introduction}

The Rosenblatt Perceptron \cite{rosenblatt1958perceptron} is the first mathematical model of a biological neuron which is designed to solve a binary classfication problem, i.e. the problem of choosing the correct class for given examples, represented as feature vectors.
The perceptron needs to be trained on some training dataset in order to determin the best hyperplane to separate them:
hopefully, the learned hyperplane will be able to correctly classify also new examples.

In this work we implement and train a Rosenblatt Perceptron and run some computer sinulations to verify its theoretical properties.
In particular, we try to estimate the capacity of the separation hyperplane learned by the perceptron.

\section{Theory}
\label{sec:fundamentals}
As briefly mentioned in the introduction the Perceptron is the first tempt to create a mathematical model of
a neuron and for that reason, exactly like a neuron, it is thought to have incoming and outgoing connections. Since it is a standalone
model the incoming connections are represented by a weights' vector (of total size $N$) and the only outgoing one consists in a value and represents
the output of the Perceptron (i.e. the result of the classification). As touched above this model has $N$ incoming connections,
where $N$ is the dimension of the labels' vector $\xi$ taken as an input (or in the biological context the information coming from the other neurons).\\
The sign of the dot product between these labels ($N$) and the vector containing the weights ($\mathsf{\bm{w}}$) minus a certain threshold
$\theta$ is computed and results in the state (active or inactive) of the perceptron (i.e. the result of the classification problem),
the complete formula is reported below (equation (\ref{perceptron-activation})).

\begin{equation} \label{perceptron-activation}
    S = sign(\mathsf{\bm{w}} \cdotp \xi) = \pm 1
\end{equation}

To train the Perceptron we implemented the Rosemblatt's perceptron algorithm (\cite{rosenblatt1958perceptron}), that consists in
presenting at every step $t$ an input vector $\xi$ to the perceptron, calculate the error as how much the classification was distant from
the correct labeling $\bm{S}$ (through the formula (\ref{perceptron-error})) and update the values of the weights' vector in case the former
calculated error is greater than zero (or in a general case $c$ as we will discuss in the bonus).

\begin{equation} \label{perceptron-error}
    E^\mu = \mathsf{\bm{w}} \cdotp \xi^\mu - S^\mu_R
\end{equation}

The weights update consists in increasing (or decreasing) their previous value by an amount proportional to the error made
by the perceptron while solving the previous classification task. The formula used to computed the values of the updated weight vectors is reported in (equation
\ref{perceptron-weight-update}).

\begin{equation} \label{perceptron-weight-update}
    \mathsf{\bm{w}}(t+1) = \mathsf{\bm{w}}(t) + \frac{1}{N} \big[c - E^{v(t)}\big] \bm{\xi}^{v(t)} S^{v(t)}_R
\end{equation}

The learning process (i.e. the presentation of the input and the update of the weights) is iterated till either the perceptron is able to classify correctly
all the data contained in the training set or it reaches a maximum number of epochs (i.e. iterations).
\section{Implementation}

\subsection{Dataset Generation}
The first step of the experiment is to generate an artificial dataset with $P$ random N-dimensional feature vectors and binary labels $\mathcal{D} = \{ \xi^\mu, S^\mu \,|\, \xi^\mu \in \mathds{R}^N, S^\mu \in \{+1, -1\} \}_{\mu=1}^P$.
The feature vectors $\xi^\mu$ have independent random components $\xi_j^\mu \sim \mathcal{N}(0, 1)$ and are generated using the \texttt{randn} command in Matlab.
The labels $S^\mu$ are assume the values $\{+1, -1\}$ with equal probability of $1/2$.

\begin{lstlisting}[language=Matlab]
  function [X, y] = generate_dataset(P, N)
      X = randn(P, N);
      y = iff(rand(P, 1) < 0.5, -1, 1);
  end
\end{lstlisting}

\subsection{Perceptron Training}
The next step is to use the Rosenblatt algorithm to implement and train a perceptron on the generated dataset.
The Rosenblatt algorithm is a sequential procedure that learns the weights of the perceptron from the examples.
The weights are initialized to zero, then the training examples are presented one by one to the perceptron for a given number of epochs (iterations).
If the example is correctly classified according to the current weights, no update is performed;
if the example if wrongly classified, the weights are updated according to the following rule:
\begin{gather*} 
    w(0) = 0, \\
    w(t + 1) =  \begin{cases}
                    w(t) + \frac{1}{N} \xi^{\mu(t)} S^{\mu(t)} &\text{if $E^{\mu(t)} \leq 0$}\\
                    w(t) &\text{else}
                \end{cases},
\end{gather*}
where $E^{\mu(t)} = w(t) \xi^{\mu(t)} S^{\mu(t)}$, $t = 1, 2, ...$ represents the current time step and $\mu(t) = 1, 2, ..., P, 1, 2, ...$ identifies the current example.
The algorithm is stopped either when $E^\mu > 0$ for all examples in the dataset or the maximum number of epochs $n_{max}$ is reached.

\subsection{Experiments}
For a fixed value of $P$ and $N$, $n_D$ independent datasets are generated.
A new perceptron is trained on each dataset for at most $n_{max}$ epochs.
For each dataset, we compute the rate fraction $Q_{l.s.}$ of successful runs, where a run is successful if the perceptron correctly classifies each example in the dataset at the end of the training phase.

Multiple experiments are run for different values of $P$ and $N$ in order to compute $Q_{l.s.}$ as a function of $\alpha = P / N$.
In other words, we study the behaviour of the perceptron as a function of the rate between the number of examples and the dimension of the input.

\section{Evaluation}
\label{sec:evaluation}

To be able to compare the results of different experiments, we have fixed the number of iterations of Stochastic Gradient Descent to a constant value of $20000$.

\subsection{Train Error}
\begin{figure}[t]
	\centering
	\includegraphics[width=\columnwidth]{figures/error}
	\caption{Train and test error of the network for $P = 1000$. The training is done with a fixed learning rate $\eta = 0.05$.}
	\label{fig:training_error}
\end{figure}

\cref{fig:training_error} shows the train and test error for the network trained using Stochastic Gradient Descent on the first $1000$ examples.
Both the train error and test error drop very quickly during approximately the first $1000$ iterations, then remain almost constant for the rest of the training.
The test error is slightly bigger than the train error:
at the end of the training, the train error is around $0.10$ and the test error around $0.12$.
Since the test error never increases, the model does not seem to overfit the training data.

\subsection{Learned Weights}
\begin{figure}[t]
	\centering
	\includegraphics[width=\columnwidth]{figures/weights_p_1000}
    \caption{Weights of the hidden units after the training for $P = 1000$.}
	\label{fig:weights}
\end{figure}

\cref{fig:weights} shows the weights learned from the hidden units of the networks after the training for $P = 1000$.
As expected, the units learn different weights thanks to the random initialization of the weights.

\subsection{Train Dataset}
\begin{figure}[t]
	\centering
	\includegraphics[width=\columnwidth]{figures/error_ps}
	\caption{Train and test error of the network for different numbers of training examples $P$. The training is done with a fixed learning rate $\eta = 0.05$.}
	\label{fig:ps}
\end{figure}

\cref{fig:ps} shows the train and test error for different values of $P$, i.e. for a network trained on training dataset of different dimensions.
For very small training datasets ($P = 20$, $P = 50$), the train error drops very quickly and stabilizes close to $0$;
the test error is high and even increases during the training for $P = 50$.
It seems like the training set is too small for the network to generalize properly to new data, so it tends to overfit the train set.

For $P = 200$, the train error is higher, but the test error gets significantly smaller.
Still, the test error increases during the training, which indicates overfitting.

For $P = 500$, the train and test error get closer and remain constant during most of the training.
The results for higher values of $P$ are very similar and are thus not shown here.

\subsection{Train Policy}
The learning rate influences the number of iterations the training algorithm needs to converge to the optimal weights.
A high learning rate causes the process to be unstable, while a low one takes too long time to converge.
For this reason, we implement different schedulers for the learning rate able to change its value over time as a function of the current iteration: \textit{fixed}, \textit{step}, \textit{exponential} and \textit{cycle}.
\cref{fig:learning_rates_policies} shows the learning rate change over time for the different strategies.
From the application of these \ac{LRPs} we expect a change in the behavior of the error function.
In most \ac{LRPs} the learning rate decreases over time, so the error should get more stable.

\begin{figure}
	\centering
	\includegraphics[width=\columnwidth]{figures/learning_rates}
	\caption{Learning rate evolution over iterations for different learning strategies.}
	\label{fig:learning_rates_policies}
\end{figure}

\cref{fig:lrp_training_error} shows that different \ac{LRPs} have different effects on the error trend.
Overall, all implemented \ac{LRPs} perform better than the \textit{fixed}($0.05$).
The next sections discuss them in more detail.

\begin{figure}
	\centering
	\includegraphics[width=\columnwidth]{figures/error_strategies}
	\caption{Train and test error for different learning strategies.}
	\label{fig:lrp_training_error}
\end{figure}

\subsubsection{Fixed \ac{LRP}}
As the name suggests this \ac{LRP} leaves the learning rate fixed during iterations.
The error to drops down after around $500$ iterations, then decreases very slowly. 
On our dataset, this strategy yields the best performances for a learning rate $\eta = 0.02$.

\subsubsection{Step \ac{LRP}}
This strategy consists in decreasing the learning rate of a defined quantity (\textit{drop}) after a certain number of iterations (\textit{step\_size}).
As shown in \cref{fig:learning_rates_policies}, we chose \textit{step\_size} = $5000$ and \textit{drop} = $0.01$.
\cref{fig:lrp_training_error} shows a drop in both train and test error close to the step in the learning rate.

\subsubsection{Exponential \ac{LRP}}
This strategy consist in decreasing the learning rate in an exponential way, i.e. in dividing it by a constant value at each iteration.
This \ac{LRP} lets the network learn quickly at the beginning, when the currently weights are likely to be far from optimal one, and then slowly, when they are getting closer to it.
\cref{fig:lrp_training_error} shows a smoother decrease of both the train and test errors, which reaches a value smaller than $0.1$ after just $3000$ iterations.

\subsubsection{Cycle}
This strategy consists in increasing and decreasing the learning rate linearly over the number of iterations, creating a sawteeth plot, as shown in \cref{fig:learning_rates_policies}.
The aim of this \ac{LRP} is to avoid local minima letting the value of the learning rate to grow again when it reaches its minimum value.
\cref{fig:lrp_training_error} shows that this strategy performs well as far as the value of the learning rate is smaller than $0.02$, then the errors increase following the increase of the learning rate.

\section{Conclusion}
\label{sec:conclusion}

Stochastic Gradient Descent is an effective algorithm to learn the parameters of a feed-forward neural network.
The algorithm is iterative and takes as an input the number of iterations and a learning rate:
it is important to choose an appropriate number of iterations and learning rate.
If the number of iterations is too small, the algorithms does not manage to learn the optimal parameters for the network;
if the number is too big, the model may overfit the training data and have bad performances on new data.
Similarly, if the learning rate is too small, the training may take too long or even stuck in local minima;
if it is too big, the updates may be too big and the model may never reach the optimal weights.

In general, it is difficult to choose the appropriate learning rate.
In some cases it may be effective to used a time-dependent one:
for example, one may set a big learning rate at the beginning, the reduce it to make convergence easier.
We discussed some possible strategies and their effect on our regression problem.
However, a complete discussion of the possible learning rate policies is outside the scope of this document.

\section{Individual Workload}
The workload of this assignment was divided as follows.

Samuel Giacomelli:
\begin{itemize}
    \item Generation of the datasets with noise.
    \item Basic implementation of the MinOver algorithm.
    \item Experiments on datasets with different degrees of noise, both for the MinOver and Rosenblatt training algorithms.
\end{itemize}

Davide Pedranz:
\begin{itemize}
    \item Generation of the datasets without noise.
    \item Stopping criteria and optimization for the MinOver training algorithm.
    \item Experiments for the MinOver algorithm and comparison with the Rosenblatt one in absence of noise in the data.
\end{itemize}

We worked together on the report.


\bibliographystyle{IEEEtran}
\bibliography{references}

\end{document}
