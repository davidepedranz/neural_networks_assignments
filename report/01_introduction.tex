\section{Introduction}
\label{sec:introduction}

In this work we will present an implementation of a Rosenblatt Perceptron \cite{rosenblatt1958perceptron}
and how we made use of it in order to obtain suitable data that let us observe how the theory about the capacity of the
hyperplane can be explained through a computer experiment. This experiment can also be interpreted as a study about how many
data can be stored in the Perceptron, even though this is not the main aim of this mathematical construct.
In fact the Rosenblatt Perceptron is the first mathematical model of a biological neuron which is designed to
solve the problem of classifying an input by finding a linear separation in the context of a set containing $P$ $N$-dimensional input vectors
and in order to be able to determine wether an element belongs to a class or not it needs to be trained on some labeled data
(i.e. it needs to go through a supervised training session).